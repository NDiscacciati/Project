\documentclass[10pt]{beamer}

\usepackage[american]{babel}
\usepackage[utf8]{inputenc}
\usepackage[T1]{fontenc}
\usepackage{lmodern}
\usepackage{amsmath,amsfonts,amssymb}
\usepackage{amsthm}
\usepackage{graphicx}
%\usepackage{geometry}
%\usepackage[top=0.6in, bottom=1in, left=0.8in, right=0.8in]{geometry}
%\geometry{a4paper}
\usepackage[parfill]{parskip}
\usepackage{graphicx}
\usepackage{amssymb}
\usepackage{epstopdf}
\usepackage{color}
%\usepackage[tt]{titlepic}
\usepackage{fancyhdr}
\usepackage{enumerate}
%\usepackage{lastpage}
\usepackage[labelformat=simple]{subcaption}
%\usepackage{subfigure}
\usepackage{caption}
\usepackage{bm}
\usepackage{verbatim}
\usepackage{float}
%\usepackage[numbered,framed]{matlab-prettifier}
%\usepackage{bigfoot}
\usepackage{afterpage}
%\usepackage[]{algorithm2e}
%\usepackage{algorithm}
%\usepackage[noend]{algpseudocode}
\usepackage{listings}
\usepackage{xcolor}
\usepackage{hyperref}
\usepackage{mathtools}




% Custom Defines
%\usepackage[comma,numbers,sort&compress]{natbib}
%\bibliographystyle{plainnat}
%\usepackage[pdfstartview=FitH,
%            breaklinks=true,
%            bookmarksopen=true,
%            bookmarksnumbered=true,
%            colorlinks=true,
%            linkcolor=black,
%            citecolor=black
%            ]{hyperref}
%\newcommand{\rmd}{\textrm{d}}
%\newcommand{\bi}[1]{{\ensuremath{\boldsymbol{#1}}}}
%\definecolor{gray}{rgb}{0.5,0.5,0.5}

%\topmargin=-0.45in      %
%\evensidemargin=0in     %
%\oddsidemargin=-0.1in      %
%\textwidth=6.8in        %
%\textheight=9.2in       %
%\headsep=0.25in         %
%\headheight=30.9pt

\graphicspath{{../Figures/}}
\renewcommand\thesubfigure{(\alph{subfigure})}
%\epstopdfsetup{outdir=./}

%\numberwithin{equation}{section}
%\numberwithin{figure}{section}

\makeatletter
\def\BState{\State\hskip-\ALG@thistlm}
\makeatother

\lstset { %
	language=C++,
	backgroundcolor=\color{black!5}, % set backgroundcolor
	basicstyle=\footnotesize,% basic font setting
}

%\let\ph\mlplaceholder % shorter macro
%\lstMakeShortInline"

%\lstset{
%	style              = Matlab-editor,
%	basicstyle         = \mlttfamily,
%	escapechar         = ",
%	mlshowsectionrules = true,
%}

\usetheme{Copenhagen}


%\let\endtitlepage\relax

\newcommand{\norm}[2]{\left\lVert#1\right\rVert_{#2}}
\newcommand{\partialdiff}[2]{\frac{\partial#1}{\partial#2}}
\newcommand{\dof}{\text{dof}}
\newcommand{\mymatrix}[1]{\mathbf{#1}}
\newtheorem{thm}{Theorem}
\newtheorem{prop}{Proposition}

\setbeamertemplate{navigation symbols}{}

\begin{document}

	\title{Virtual Element Methods for elliptic problems}  
	\author{Niccolò Discacciati}
	\institute[PoliMi]{Project for the course "Numerical analysis for partial differential equations" \\Politecnico di Milano}
	\date{March 12, 2018} 
	
	\begin{frame}
		\titlepage
	\end{frame}
	
	\begin{frame}\frametitle{Table of contents}
		\tableofcontents
	\end{frame} 
	
	
	\section{Introduction} 
	
	\begin{frame}\frametitle{Strengths of the VEM} 
		\begin{itemize}
		\item Combine features from different classes of solvers (FV, MFD, FEM, \dots) \\
		\item Non-polynomial basis and test functions (but exact computations with polynomials) \\
		\item Generic grids (generic shapes, non-convex polygons, \dots)
		\end{itemize}
	\end{frame}
	\begin{frame} \frametitle{Project description}
		We provide a C++ implementation for the 2D VEM for\\
		\vspace{0.5cm}
		Laplace problem:
		\begin{equation*}
		\begin{cases}
		-\Delta u = f & \mbox{in } \Omega \\
		u = g & \mbox{on } \partial \Omega
		\end{cases}
		\end{equation*}
		
		Transport-diffusion problem:
		\begin{equation*}
		\begin{cases}
		-\nabla \cdot (\mu \nabla u)+\beta \cdot \nabla u = f & \mbox{in } \Omega \\
		u = g & \mbox{on } \partial \Omega
		\end{cases}
		\end{equation*}
		
		Continuous pb: find $u \in V$ s.t. $a(u,v) = \langle f,v \rangle \quad \forall v \in V$ \\
		Discrete pb: find $u_h \in V_h$ s.t. $a_h(u_h,v_h) = \langle f_h,v_h \rangle \quad \forall v_h \in V_h $ \\
	\end{frame}
	
	
	\section{Theoretical background} 
	\subsection{Building blocks}
	\begin{frame}\frametitle{Abstract framework}
	\begin{block}{Abstract Convergence Theorem}
		Assume that:
		\begin{itemize}
			\item \dots
			\item we define a symmetric bilinear form $a_h$ and a local one $a_h^K$ such that
			$$a_h(u_h,v_h)=\sum_{K \in \mathcal{T}_h} a_h^K(u_h,v_h)$$
			\item there exists an integer $k\geq 1$ such that $\mathbb{P}_k(K) \subset V_h|_K$ and
			\begin{itemize}
				\item \textit{k-consistency}: for all $p \in \mathbb{P}_k(K)$, $v_h \in V_h|_K$, we have $a_h^K(p,v_h)=a^K(p,v_h)$
				\item\textit{stability}:
				$\alpha_{*}a^K(v_h,v_h) \leq a_h^K(v_h,v_h) \leq \alpha^{*}a^K(v_h,v_h)$
			\end{itemize}
		\end{itemize}
	\vspace{0.25cm}
	Then, the discrete problem has a unique solution.
	Moreover, for every approximation $u_I \in V_h$, $u_\pi$ that is piecewise in $\mathbb{P}_k(\Omega)$
	$$\norm{u-u_h}{H_0^1(\Omega)} \leq C \left( 
	\norm{u-u_I}{H_0^1(\Omega)}+\norm{u-u_\pi}{H_0^1(\Omega)}+\norm{f-f_h}{V'_h} \right)
	$$
	\end{block}
	\end{frame}
	
	\begin{frame} \frametitle{Local discrete space}
		Define, for $k \geq 1$ and a simple polygon $K$, the spaces
		\begin{equation*}
		\mathbb{B}_k(\partial K) \coloneqq \lbrace v \in C^0(\partial K) \, : \, v_{|_e} \in \mathbb{P}_k(e), \, \forall e \subset \partial K \rbrace
		\end{equation*}
		\begin{equation*}
		V^{K,k} \coloneqq \lbrace v \in H^1(K) \, : v_{|\partial K} \in \mathbb{B}_k(\partial K), \, \Delta v |_K \in \mathbb{P}_{k-2}(K) \rbrace
		\label{eqn:VKk}
		\end{equation*}
		Remarks: 
		\begin{itemize}
			\item $\mathbb{P}_k(K) \subset V^{K,k}$
			\item dim $V^{K,k}= n k + \frac{k(k-1)}{2}$, where $n=\#$ vertices
		\end{itemize}
	\end{frame}	

	\begin{frame} \frametitle{Degrees of freedom}
		Degrees of freedom:
		\begin{itemize}
			\item $\mathcal{V}^{K,k}$: the value of $v_h$ at the vertices of the polygon.
			\item $\mathcal{E}^{K,k} $: for $k>1$, on each edge $e$, the value of $v_h$ at the $k-1$ internal points of the $k+1$ Gauss-Lobatto quadrature rule on $e$.
			\item $\mathcal{P}^{K,k}$: for $k>1$, the moments
			$$\frac{1}{|K|} \int p(\mathbf{x})v_h(\mathbf{x})d\mathbf{x} \quad \forall p(\mathbf{x}) \in \mathcal{M}_{k-2}(K)$$
			where $\mathcal{M}_k(K) \coloneqq \bigg\lbrace \left( \frac{\mathbf{x}-\mathbf{x}_K}{h_K} \right)^\mathbf{s}, |\mathbf{s}|\leq k \bigg\rbrace$
		\end{itemize}
		\begin{theorem}
			The degrees of freedom $\mathcal{V}^{K,k} \cup \mathcal{E}^{K,k} \cup \mathcal{P}^{K,k}$ are unisolvent for $V^{K,k}$.
		\end{theorem}
		Remark: this choice for the dofs is not unique.
	\end{frame}
	
%	\begin{frame}\frametitle{Global discrete space}
%Recall that: $V^{K,k} \coloneqq \lbrace v \in H^1(K) \, : v_{|\partial K} \in \mathbb{B}_k(\partial K), \, \Delta v |_K \in \mathbb{P}_{k-2}(K) \rbrace$ \\
%\vspace{0.5cm}
%Thus:
%\begin{equation*}
%V_h \coloneqq \lbrace v \in H^1_0(\Omega) \ : \ v_{|\partial K} \in \mathbb{B}_k(\partial K), \, \Delta v_{|K} \in \mathbb{P}_{k-2}(K), \quad \forall K \in \mathcal{T}_h \rbrace
%\end{equation*}
%Remark: dim $V_h= N_V + N_E (k-1) + N_P \frac{k(k-1)}{2}$, where $N_V, \ N_E, \text{ and} \ N_P$ denote the total number of internal vertices, internal edges and elements respectively. \vspace{0.5cm}

%Degrees of freedom: same as the local ones, taking into account the Dirichlet conditions.
%	\end{frame}
	
	\subsection{Projection operators}
	\begin{frame}\frametitle{The projection operator $\Pi^\nabla$}
Define $\Pi_k^\nabla : V^{K,k} \rightarrow \mathbb{P}_k(K)$ as
\begin{equation*}
\begin{cases}
	\int_K \nabla \Pi_k^\nabla v_h \cdot \nabla q = \int_K \nabla v_h \cdot \nabla q, \quad \forall q \in \mathbb{P}_k(K) \\
	P_0(\Pi^\nabla v_h) = P_0(v_h)
\end{cases}
\label{eqn:pi_nabla}
\end{equation*}
where $P_0$ is defined as
	\begin{align*}
	P_0 v_h &\coloneqq \frac{1}{n} \sum_{i=1}^{n} v_h(V_i) \quad \text{if $k=1$ ($\lbrace V_i \rbrace_{i=1}^n$ is the set of vertices)} 
	\\
	P_0 v_h &\coloneqq \frac{1}{|K|} \int_{K} v_h \quad \text{if $k\geq 2$} 
	\end{align*}
%\begin{theorem}
%	The operator $\Pi_k^\nabla$ is an orthogonal projection operator. In particular, for a polynomial $q \in \mathbb{P}_k(K)$ we have $\Pi_k^\nabla q = q$.
%	\label{prop:pi_nabla}
%\end{theorem}
Remark: it can be computed using only the dofs of $v_h$. 
	\end{frame}
	
	\begin{frame}\frametitle{The $L^2$-projection operator $\Pi^0$}
Define $\Pi^0_k: V^{K,k} \rightarrow \mathbb{P}_k(K)$ as
\begin{equation*}
\int_K \Pi_k^0 v_h \ q = \int_K v_h \ q, \quad \forall q \in \mathbb{P}_k(K)
\label{eqn:pi_0}
\end{equation*}
%\begin{theorem}
%	The operator $\Pi_k^0$ is an orthogonal projection. In particular, for a polynomial $q \in \mathbb{P}_k(K)$ we have $\Pi_k^0 q = q$.
%\end{theorem}
Remark: not computable for polynomials of degree $k-1, \ k$. \\
Idea: replace $v_h$ with $\Pi_k^\nabla v_h$ when such polynomials are involved. \\ 
\begin{equation*}
\int_K \Pi_k^0 v_h \ m_\alpha = 
\begin{cases}
\int_K v_h \ m_\alpha &\mbox{ if deg($m_\alpha$) $\leq$ $k-2$} \\
\int_K \Pi_k^\nabla v_h \ m_\alpha &\mbox{otherwise}
\end{cases}
\end{equation*}
Remark: we are not introducing errors (modified VEM space).
	\end{frame}
	
	\subsection{Concrete definitions}

\begin{frame} \frametitle{Discrete bilinear form $a_h$}
Set
	\begin{equation*}
	a_h^K(u,v) \coloneqq a^K(\Pi_k^\nabla u, \Pi_k^\nabla v)+S^K(u-\Pi_k^\nabla u, v-\Pi_k^\nabla v) \quad \forall u,v \in V^{K,k}
	\label{eqn:ah}
	\end{equation*}	
	where $S^K(u,v)$ is a symmetric positive definite bilinear form such that
	\begin{equation*}
	c_1 \ a^K(v,v) \leq S^K(v,v) \leq c_2 \ a^K(v,v), \quad \forall v \in V^{K,k} \, \text{with } \Pi_k^\nabla v =0
	\label{eqn:Sk}
	\end{equation*}
	
	\begin{theorem}
		The bilinear form $a_h^K$ satisfies the $k$-consistency and the stability properties.
	\end{theorem}
Motivated by scaling arguments, we define
\begin{equation*}
S^K(u,v)=\sum_{r=1}^{N^K} \dof_r(u) \dof_r(v)
\end{equation*}
\end{frame}	
	
%\begin{frame} \frametitle{Discrete bilinear form $a_h$}
%	Let $S^K(u,v)$ be a symmetric positive definite bilinear form such that
%	\begin{equation*}
%	c_1 \ a^K(v,v) \leq S^K(v,v) \leq c_2 \ a^K(v,v), \quad \forall v \in V^{K,k} \, \text{with } \Pi_k^\nabla v =0
%	\label{eqn:Sk}
%	\end{equation*}
%	for some constants $c_1,c_2>0$ independent of $K$ and $h_K$. \\
%	Set
%	\begin{equation*}
%	a_h^K(u,v) \coloneqq a^K(\Pi_k^\nabla u, \Pi_k^\nabla v)+S^K(u-\Pi_k^\nabla u, v-\Pi_k^\nabla v) \quad \forall u,v \in V^{K,k}
%	\label{eqn:ah}
%	\end{equation*}	
%	\begin{theorem}
%		The bilinear form $a_h^K$ satisfies the $k$-consistency and the stability properties.
%	\end{theorem}
%\end{frame}

%\begin{frame} \frametitle{Discrete bilinear form $a_h$}
%Motivated by scaling arguments, we define
%	\begin{equation*}
%	S^K(u,v)=\sum_{r=1}^{N^K} \dof_r(u) \dof_r(v)
%	\end{equation*}
%	and finally:
%	\begin{equation*}
%	a_h^K(u,v) \coloneqq a^K(\Pi_k^\nabla u, \Pi_k^\nabla v)+\sum_{r=1}^{N^K} \dof_r(u-\Pi_k^\nabla u) \ \dof_r(v-\Pi_k^\nabla v) \quad \forall u,v \in V^{K,k}
%	\end{equation*}
%\end{frame}

\begin{frame} \frametitle{Discrete load term $f_h$}
	Multiple choices are can be made for the approximation of the right-hand-side. \\ 
	Idea: guarantee optimal error estimates in suitable norms. \\
	\vspace{0.5cm}
	\textit{Safe} choice:
	\begin{equation*}
	\langle f_h,v \rangle= \langle \Pi_{k}^0 f, v \rangle = \int_K f \ \Pi_{k}^0 v
	\label{eqn:RHS_impl}
	\end{equation*}
\end{frame}

\begin{frame} \frametitle{Error estimates}
	Preliminary results:
	\begin{itemize}
		\item Projection error: under suitable assumptions, there exists $w_\pi$ s.t.
		$$\norm{w-w_\pi}{0,K}+h_K |w-w_\pi|_{1,K} \leq C h_K^s |w|_{s,K}$$
		\item Interpolation error: under suitable assumptions, there exists $w_I$ such that
		$$\norm{w-w_I}{0,K}+h_K |w-w_I|_{1,K} \leq C h_K^s |w|_{s,K}$$
	\end{itemize}
	\begin{block}{Error estimates}
		Under suitable regularity assumptions on $u$ and $f$, with the above-mentioned definitions for $a_h$ and $f_h$ we have
		$$\norm{u-u_h}{L^2(\Omega)} +h \norm{u-u_h}{H_0^1(\Omega)} \leq C h^{k+1}$$ for a constant $C=C(u,f)>0$ independent of $h$.
	\end{block}
\end{frame}
	
	\subsection{Extension to elliptic problems}
	\begin{frame} \frametitle{Extension to elliptic problems}
		
With previous definition of $a_h$ we have loss of optimal convergence rate for non-constant coefficients.

Consistency part:
	\begin{align*}
	\int_K \mu \nabla u \cdot \nabla v &\simeq \int_K \mu (\Pi^0_{k-1} \nabla u) (\Pi^0_{k-1} \nabla v)& \quad \text{(diffusion part)} 
	\\
	\int_K (\beta \cdot \nabla u) \cdot v &\simeq \int_K \beta (\Pi^0_{k-1} \nabla u) (\Pi^0_{k} v) & \quad \text{(transport part)} \\
	\int_K \alpha u v &\simeq \int_K \alpha (\Pi^0_{k} u) (\Pi^0_{k} v) & \quad \text{(reaction part)} 
	\end{align*}
%Stability part: no significant changes, usually done only for diffusive part
%\begin{equation*}
%S^K(u,v)=\bar{\mu} \sum_{r=1}^{N^K} \dof_r(u) \dof_r(v)
%\label{eqn:Sk_weighted_def}
%\end{equation*}
%where $\bar{\mu}$ denotes a constant approximation of $\mu$. \\
\end{frame}

\begin{frame} \frametitle{Extension to elliptic problems}
Stability part: no significant changes, usually done only for diffusive part
\begin{equation*}
S^K(u,v)=\bar{\mu} \sum_{r=1}^{N^K} \dof_r(u) \dof_r(v)
\label{eqn:Sk_weighted_def}
\end{equation*}
where $\bar{\mu}$ denotes a constant approximation of $\mu$. 
\vspace{0.5cm}

Load term: no significant changes.

Error estimates: same as Laplace problem.
%	\begin{block}{Error estimates}
%		Under suitable regularity assumptions on the data, with the above-mentioned definitions for $a_h$ and $f_h$ we have
%		$$\norm{u-u_h}{L^2(\Omega)} +h \norm{u-u_h}{H_0^1(\Omega)} \leq C h^{k+1}$$ for a constant $C=C(u,f,\mu,\beta)>0$ independent of $h$.
%	\end{block}
\end{frame}
	
	
	\section{Numerical results} 
	\begin{frame}\frametitle{Numerical results}
Validation is performed thanks to:
	\begin{itemize}
	\item the scheme being exact for polynomials up to degree $k$
	\item computation of the discretization error
	\begin{itemize}
		\item $H^1_0$-norm: $\norm{u-u_h}{H_0^1(\Omega)}=\norm{\nabla u-\nabla u_h}{L^2(\Omega)}=\left( (\mathbf{u}-\mathbf{u_h})^T \mymatrix{A} (\mathbf{u}-\mathbf{u_h}) \right)^{1/2}$
		\item $L^2$-norm: $\norm{u-u_h}{L^2(\Omega)}=\left( (\mathbf{u}-\mathbf{u_h})^T \mymatrix{M} (\mathbf{u}-\mathbf{u_h}) \right)^{1/2}$
		\item $L^\infty$-norm: $\norm{u-u_h}{L^\infty(\Omega)}=\max \lbrace | \hat{\mathbf{u}}-\hat{\mathbf{u}}_\mathbf{h} |\rbrace$ (only boundary dofs)
	\end{itemize}
\end{itemize}
\end{frame}


\begin{frame} \frametitle{Meshes}

Domain: $\Omega=[0,1]^2$.
Meshes: generated with PolyMesher.

\begin{figure}[H]
	\centering
	\begin{subfigure}{0.49\textwidth}
		\centering
		\includegraphics[width=\linewidth]{uniform_grid.eps}
		\subcaption{Structured mesh}
		\label{fig:uniform}
	\end{subfigure}
	\begin{subfigure}{0.49\textwidth}
		\centering
		\includegraphics[width=\linewidth]{nonuniform_grid.eps}
		\subcaption{Voronoi mesh}
		\label{fig:voronoi}
	\end{subfigure}
	\caption{Example of the two mesh types used to validate the scheme}
	\label{fig:meshes}
\end{figure} 

	\end{frame}
	
	\subsection{Laplace problem}
	
\begin{frame} \frametitle{Laplace problem, $k=1$}
	
\begin{figure}[H]
	\centering
	\begin{subfigure}{0.49\textwidth}
		\centering
		\includegraphics[width=\linewidth]{k1_64elem_poly.eps}
		\subcaption{$u(x,y)=x+y$}
		\label{fig:polynomial}
	\end{subfigure}
	\begin{subfigure}{0.49\textwidth}
		\centering
		\includegraphics[width=\linewidth]{k1_64elem_sin.eps}
		\subcaption{$u(x,y)=\sin(2\pi x) \sin(2 \pi y)$}
		\label{fig:sinsin}
	\end{subfigure}
	\caption{Contour plot of the discrete solution, generated with uniform meshes consisting of $N^2=64^2$ elements}
\end{figure} 
	
\end{frame}

\begin{frame} \frametitle{Laplace problem, $k=1$, uniform grid}
	Exact solution: $u(x,y)=\sin(2\pi x) \sin(2 \pi y)$ 
\begin{figure}[H]
	\centering
	\begin{subfigure}{0.32\textwidth}
		\centering
		\includegraphics[width=\linewidth]{k1_H1_uniform.eps}
		\subcaption{$H^1_0$-norm}
		\label{fig:k1_H1_uniform}
	\end{subfigure}
	\begin{subfigure}{0.32\textwidth}
		\centering
		\includegraphics[width=\linewidth]{k1_L2_uniform.eps}
		\subcaption{$L^2$-norm}
		\label{fig:k1_L2_uniform}
	\end{subfigure}
	\begin{subfigure}{0.32\textwidth}
		\centering
		\includegraphics[width=\linewidth]{k1_infinity_uniform.eps}
		\subcaption{$L^\infty$-norm}
		\label{fig:k1_infinity_uniform}
	\end{subfigure}
		\caption{Discretization error under $h$-refinement, $k=1$, structured grids.}
	\label{fig:k1_uniform}
\end{figure}
	Superconvergence in $H_0^1$ and $L^\infty$ norms due to solution smoothness and structured mesh. \\
\end{frame}

\begin{frame} \frametitle{Laplace problem, $k=1$, nonuniform grid}
	Exact solution: $u(x,y)=\sin(2\pi x) \sin(2 \pi y)$
\begin{figure}[H]
	\centering
	\begin{subfigure}{0.32\textwidth}
		\centering
		\includegraphics[width=\linewidth]{k1_H1_nonuniform.eps}
		\subcaption{$H^1_0$-norm}
		\label{fig:k1_H1_nonuniform}
	\end{subfigure}
	\begin{subfigure}{0.32\textwidth}
		\centering
		\includegraphics[width=\linewidth]{k1_L2_nonuniform.eps}
		\subcaption{$L^2$-norm}
		\label{fig:k1_L2_nonuniform}
	\end{subfigure}
	\begin{subfigure}{0.32\textwidth}
		\centering
		\includegraphics[width=\linewidth]{k1_infinity_nonuniform.eps}
		\subcaption{$L^\infty$-norm}
		\label{fig:k1_infinity_nonuniform}
	\end{subfigure}
	\caption{Discretization error under $h$-refinement, $k=1$, Voronoi grids.}
	\label{fig:k1_nonuniform}
\end{figure}
	Loss of superconvergence but remains coherent with the theoretical results.
\end{frame}


\begin{frame} \frametitle{Laplace problem, $k=4$, uniform grid}
	Exact solution: $u(x,y)=\sin(2\pi x) \sin(2 \pi y)$
\begin{figure}[H]
	\centering
	\begin{subfigure}{0.32\textwidth}
		\centering
		\includegraphics[width=\linewidth]{k4_H1_uniform.eps}
		\subcaption{$H^1_0$-norm}
		\label{fig:k4_H1_uniform}
	\end{subfigure}
	\begin{subfigure}{0.32\textwidth}
		\centering
		\includegraphics[width=\linewidth]{k4_L2_uniform.eps}
		\subcaption{$L^2$-norm}
		\label{fig:k4_L2_uniform}
	\end{subfigure}
	\begin{subfigure}{0.32\textwidth}
		\centering
		\includegraphics[width=\linewidth]{k4_infinity_uniform.eps}
		\subcaption{$L^\infty$-norm}
		\label{fig:k4_infinity_uniform}
	\end{subfigure}
	\caption{Discretization error under $h$-refinement, $k=4$, structured grids.}
	\label{fig:k4_uniform}
\end{figure}
	Superconvergence is still present, and becomes even more evident. Using Voronoi meshes, its effect is reduced. \\
	Remark: memory issues for small $h$.
\end{frame}


\subsection{Transport-diffusion problem}


\begin{frame} \frametitle{Transport-diffusion problem, $k=4$, uniform grid}
	Exact solution: $u(x,y)=\sin(2\pi x) \sin(2 \pi y)$ \\
	Coefficients: $\mu=x+y+1$, $\beta=(1,1)^T$
\begin{figure}[H]
	\centering
	\begin{subfigure}{0.32\textwidth}
		\centering
		\includegraphics[width=\linewidth]{elliptic_k4_H1_uniform.eps}
		\subcaption{$H^1_0$-norm}
		\label{fig:ell_k4_H1_uniform}
	\end{subfigure}
	\begin{subfigure}{0.32\textwidth}
		\centering
		\includegraphics[width=\linewidth]{elliptic_k4_L2_uniform.eps}
		\subcaption{$L^2$-norm}
		\label{fig:ell_k4_L2_uniform}
	\end{subfigure}
	\begin{subfigure}{0.32\textwidth}
		\centering
		\includegraphics[width=\linewidth]{elliptic_k4_infinity_uniform.eps}
		\subcaption{$L^\infty$-norm}
		\label{fig:ell_k4_infinity_uniform}
	\end{subfigure}
	\caption{Discretization error under $h$-refinement, $k=4$, structured grids.}
	\label{fig:ell_k4_uniform}
\end{figure}
	Similar comments with respect to Laplace problem. Polynomially exact, superconvergence with uniform meshes, errors with similar order of magnitude.
\end{frame}


\section{Conclusion}
\begin{frame} \frametitle{Conclusion}

\begin{itemize}
	\item Main blocks of the VEM are the projection operators and the usage of non polynomial basis functions.
	\item VEM are well suited for polygonal meshes, with optimal convergence rates and they are polynomially exact. Superconvergence is present with structured meshes.
	\item Extensions: other boundary conditions, time-dependent problems, 3D, not second-order problems, \dots
\end{itemize}

\end{frame}

\begin{frame}
	\frametitle{Bibliography}
	\begin{thebibliography}{}
		\scriptsize {
		\bibitem{Basic_principles} L. Beir\~{a}o da Veiga, F. Brezzi, A. Cangiani, G. Manzini, L. D. Marini, A. Russo. \textit{Basic principles of Virtual Element Methods}. Mathematical Models and Methods in Applied Sciences, 2013. 
		\bibitem{hitchhiker} L. Beir\~{a}o da Veiga, F. Brezzi, L.D. Marini, A. Russo. \textit{The hitchhiker’s guide to the virtual element method}. Mathematical Models and Methods in Applied Sciences, 2014.
		\bibitem{Equivalent_proj} B. Ahmed, A. Alsaedi, F. Brezzi, L. D. Marini, A. Russo. \textit{Equivalent projectors for Virtual Element Methods}. Comput. Math. Appl., 2013.
		\bibitem{General} L. Beir\~{a}o da Veiga, F. Brezzi, L.D. Marini, A. Russo. \textit{Virtual Element Methods for general second order elliptic problems on polygonal meshes}. Mathematical Models and Methods in Applied Sciences, 2016.
		\bibitem{GeneralOnline} \url{http://www.poems15.gatech.edu/sites/default/files/talks/Russo-slides.pdf}. 2015.
		\bibitem{polymesher} C. Talischi, G. H. Paulino, A. Pereira, I. F. M. Menezes. \textit{PolyMesher: a general-purpose mesh generator for polygonal elements written in Matlab}. Struct. Multidisc. Optim., 2012.
	}
	\end{thebibliography}
\end{frame}


\end{document}  

